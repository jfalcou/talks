\documentclass[color=option]{beamer}

\mode<presentation>
\usepackage[english,french]{babel} 
\usepackage[latin1]{inputenc}
\usepackage[T1]{fontenc}
\usepackage{times} 
\usepackage{multimedia,pgf,xmpmulti,listings}
\usepackage{amsmath,amssymb,latexsym}
\usetheme{progressbar}

\definecolor{orangeish}{rgb}{1,0.45,0}
\definecolor{ForestGreen}{rgb}{.25,0.45,0.25}

\progressbaroptions{headline=sections,frametitle=normal,titlepage=normal}

%%%%%%%%%%%%%%%%%%%%%%%%%%%%%%%%%%%%%%%%%%%%%%%%%%%%%%%%%%%%%%%%%%%%%%%%
%%% Document
%%%%%%%%%%%%%%%%%%%%%%%%%%%%%%%%%%%%%%%%%%%%%%%%%%%%%%%%%%%%%%%%%%%%%%%%
\title[Boost'Con 2011]{Getting Jiggy with Proto\\
Hands-on Tutorial on C++ EDSL Design}
\author{Joel Falcou}
\institute{LRI, University Paris Sud XI}
\date{05/17/2011}

\begin{document}
\frame{\titlepage}
\selectlanguage{english}

\frame
{
  \frametitle{Why a Proto Hands on ?}
  \begin{block}{Proto is impressive}<1->
  \footnotesize
  \begin{itemize}
  \footnotesize
  \item EDSL are not easy to grasp
  \item Proto helps but is full of gotcha
  \item Few (like 10 ?) peoples are actually up to the details
  \end{itemize}
  \end{block}

  \begin{block}{Our goals}<2->
  \footnotesize
  \begin{itemize}
  \footnotesize
  \item Take a slow trip to Proto usage
  \item Go further than the simple example to see the width of it
  \item Go home with something to think about
  \end{itemize}
  \end{block}
}

\frame
{
  \frametitle{Why a Proto Hands on ?}
  \begin{block}{Schedule}<1->
  \footnotesize
  \begin{itemize}
  \footnotesize
  \item Introduction to Proto Basic : analytical functions
  \item First hand-on : evaluation of functions
  \item Second Hand-on : analytical derivative and partial derivative
  \end{itemize}
  \end{block}

  \begin{block}{How will it work ?}<2->
  \footnotesize
  \begin{itemize}
  \footnotesize
  \item Fetch the source from http://tinyurl.com/proto-hands-on
  \item Get the latest boost release or trunk
  \item Get a compiler (duh)
  \item Opens the introduction/ folder files and let's start !
  \end{itemize}
  \end{block}
}

\end{document}
